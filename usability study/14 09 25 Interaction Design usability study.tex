Steps for testing:
1. Ask for age. Record gender.
2. Ask if they have used Apple Maps. For how long and how often. Repeat for Google Maps. Test them with one(s) with which they are familiar.
3. TASK 1: Give them a screen of Los Angeles. Start timing immediately. Tell them to drop a pin anywhere on the map and get driving directions to that pin. Record errors (errors are defined as any buttons that don't get them closer to accomplishing the task).
4. Take phone, remove pin, and center on current location.
5. TASK 2: Tell the user to find the nearest Vons. Start timing immediately and record errors. Make it clear that they don't need to acquire directions to the Vons in question, they only need to find it.
6. Take phone, remove previous search inquiries, center on current location.
7. TASK 3: Find fastest (in terms of time) route—i.e., the fastest ETA—from current location to LACMA.
8. Satisfaction survey: tell users to rate on a scale of 1 to 10 how much they enjoyed performing each task on their respective devices or systems.
DATA
Apple
Test 1 average time: 23.13 seconds
Test 1 average errors: 1 error
Test 2 average time: 23.62 seconds
Test 2 average errors: 0.5 errors 
Test 3 average time: 36 seconds
Test 3 average errors: 0.92 errors
Overall satisfaction average: 6.18
Google
Test 1 average time: 19.36 seconds
Test 1 average errors: 0.59 errors
Test 2 average time: 20.53 seconds
Test 2 average errors: 0.71 errors 
Test 3 average time: 15.63 seconds
Test 3 average errors: 0.47 errors
Overall satisfaction average: 7.72
Metrics
Priority: Satisfaction, efficiency, errors
I believe the overall satisfaction the user derives from the use of the product is most important because it is a direct reflection of the experience the user goes through while utilizing the program. I don’t believe any user would rate a program highly if it was slow (inefficient) and difficult (prone to errors). I gave efficiency the second priority as it represents the ends—the goal—of the task at hand. To me, accomplishing the desired task in the least amount of time is more important than process of accomplishing the desired task. Finally I give errors the third priority as it is the most volatile of the metrics and it has the least bearing on the ends of the process. If an error is made by the user, it often will not be made again in close proximity.
With regards to efficiency, errors, and satisfaction, Google out-performed Apple in every instance except test 2 of errors. It seems natural, then, that Google Maps should be considered the better performer.
The first test we gave involved dropping a pin at any location and getting directions to it. This was an interesting task as many users, though familiar with the application itself, were not necessarily familiar with this particular function. Google performed better than Apple in both the efficiency and error metrics for this particular task. I believe this is simply because of the way Google approached the idea of dropping a pin. Apple takes a divergent approach, allowing users to drop a pin using a small "i" symbol at the bottom right of the screen that brings up a series of options including "drop a pin" as well as a quick command that involves tapping and holding a particular spot on the screen to drop a pin. Google, however, doesn’t explicitly offer a drop down menu of any sorts that allows a user to select a drop pin option: it relies completely on the hold to drop command. Though this may seem like a disadvantage, Google does something a bit differently than Apple: speed. The time one spends holding their finger down on the screen to drop a pin is significantly less in Google than it is in Apple Maps. Additionally, it is also faster to remove the pin (simply tap any other location on the map). To me, it seems as if Google did this on purpose to, not only fit within the bounds of the mental model of the users, but also set users up to accidently discover the drop pin feature. One could easily leave their finger held down briefly for a different reason and accidently drop a pin, which is where Google's quick and intuitive pin removal command comes into play. Google seems to deliberately make their application more prone to "errors" but makes them easily reversible and productive for the user with regards to his/her learning. Apple, however, takes significantly longer to drop a pin once one's finger is held down and requires the use of a drop down menu to remove the pin. This means the action has to be much more deliberate and more difficult to reverse. One could argue that an error prone application—so long as the common potential errors are allowed intelligently—can lead to a more efficient application.
	 I've noticed, too, that this appeals to the mental model of the users in question. Dropping a pin would require picking a specific location on the map, so the two most logical ways to accomplish this are to somehow have a pin dropped in the center of the map or to pick a specific location through touch. Most users began trying to accomplish this task following the second train of thought. Without explicit specifications such as a drop down menu, users began experimenting with different ways of touching: double tapping, touching with multiple fingers, or holding down the screen. Once they did, they usually got the desired result fairly quickly. Part of the mental model of pin dropping, it seems, is the timing with which the pin is dropped and Google has replicated this much more closely than Apple has. Apple users sometimes tried holding down their fingers and let go thinking that it didn't work.
	For the second test we had users to find the nearest Vons grocery store. The major difference here seemed more closely tied to the GUI itself. Apple opted for a small search bar at the top of the screen that displays a drop down search menu once it is used. Google utilized a larger search bar that extends lower on the screen and even expands once it is being utilized with its drop down menu. This doesn’t necessarily protean to  the application's ability to search but it does make Google's program much easier to use within a touch interface and therefore faster.
	The third test required users to find LACMA (Los Angeles County Museum of Art) on their map through a quick search. In this test, Google performed significantly better than Apple. Aside from the aforementioned differences in the search bars, Apple maps had a tendency to find other related locations rather than the museum itself. In my personal observations, I noticed that Google took users straight to the museum itself with no other pins. Apple, however, dropped multiple pins in the area. These pins seemed to be dropped in shops and restaurants that were near the museum or related to it somehow, yet none of the pins fell on the museum itself. The museum is a noted location on Apple Maps, meaning if one were to see it on the map, they could tap it and get directions and info for it without needing the address or to type any information. Nevertheless, this is still an extra step for the user if the search only takes them to the general area and also counts as an error. The main issue here appears that Apple Maps was not aware of the colloquial acronym people commonly use to reference the museum (I ran a personal test to very this and it did indeed perform as intended when the user typed the full name of the museum) and Google was. This falls back into the category of the mental model users expect. They expected the search engine to be smart enough to recognize commonly used acronyms and, in the case of Apple Maps, it did not. It goes without saying that the above metrics contributed to Google receiving an overall higher score than Apple when user satisfaction was recorded. 



